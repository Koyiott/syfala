\section{The \syfala team objectives}
The \syfala project ({\em Synthetiseur Faible Latence pour FPGA}) has started as a FIL project, it will probably continue for a while. This docment explains the technical choices that have been made on the first versions of the \syfala toolchain.

The \syfala toolchain is a compilation toolchain of Faust program on FPGA (currently Xilinx Zynq present on Zybo-Z7-10 board). The installation of the toolchain itself is explained  in Annex here (from p~\pageref{Annex1}).

The objective is to compile Faust\footnote{\url{https://faust.grame.fr/}} programs on a FPGA platform with the objective of obtaining a short latency between input and output of the signal.

Audio signal is sampled at (say) 48kHz. Hence one audio sample (i.e. one on each channel, two channels for stereo audio) arrives roughly every $2.083 \times 10^{-5}$ seconds, hence approximately every 20$\mu s$. In general it is considered that the latency (i.e. the time between the input of a sample and its effect on output) cannot go below 1 sample delay (i.e. 20$\mu s$). Our current syfala version is able to reach a latency of 191 $\mu s$ with the intergrated Analog Device SSM2603 codec and a latency 11.1 $\mu s$ with a more efficient codec (Analog Device ADAU 1787).

%% When performing audio processing with a software system, such as on Linux OS, the sound processing is performed by the audio driver which handles the samples coming from the audio codec. The typical application on these systems will  play music files or apply an effect on a stream. Ultra-low latency is usually not a problem on this kind of software, but efficiency is. Efficiency is needed for audio real time (computing at least on sample every 20$\mu s$) and for having as few CPU cycles as possible, as audio processing is usually sharing the CPU resources  with many other tasks.

%% For efficiency reason, all audio drivers are using buffers to communicate with the audio codec, it means that one driver activation will compute a bunch of samples, usually 64 or more. If a buffer of 64 samples is used, the latency is at least $1.3m s$ (i.e  roughly 64*20$\mu s$), then one have to add the time for interruption handling and audio processing itself. This latency has to be added to the codec latency itself (which usually can be configured, but is not negligible), which makes software solution inaplicable for low latency applications.

%% Using and FPGA to realize audio processing would take advantage of $(i)$ high computing power (parallelism can be quite high on a FPGA circuit), and  more important $(ii)$ low latency (as samples are directly coming from/going to  audio codec to/from the FPGA circuit).

Few examples of professional FPGA-based real-time audio DSP systems (i.e., Antelope Audio,\footnote{\url{https://en.antelopeaudio.com}} Korora Audio,\footnote{\url{https://www.kororaaudio.com}} etc.) and in these applications, FPGAs are dedicated to a specific task, limiting creativity and flexibility. Moreover, these designs where realized ``by hand'' i.e. by register transfer level design (in VHDL or Verilog) of the realized circuits. The idea of the \syfala project is to {\em compile} an FPGA configuration from a Faust audio processing specification. This is made possible by {\em High Level synthesis} (HLS) which is a compilation flow that transforms a software code (usually based on C-like syntax) into a HDL representation that can be further compiled with classical FPGA programming suites. The most well known HLS tools are {\tt vivadoHLS} (from {\tt Xilinx}), {\tt C2H} (from {\tt Altera}), {\tt CatapultC} (from {\tt Mentor Graphics}), but other tools are proposed today to bridge the gap between algorithmic representation and hardware level representation of a computation\footnote{See~\url{https://en.wikipedia.org/wiki/High-level_synthesis} for instance}. 
This project has been launched by the Emeraude team\footnote{\url{https://team.inria.fr/emeraude/admin}} which is  a collaboration between Grame research department\footnote{\url{https://www.grame.fr/recherche}} and Citi laboratory.
