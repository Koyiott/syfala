\section{Installation instruction of  syfala v7 toolchain}
\label{annex}
\label{install}
The Syfala toolchain is a compilation toolchain of Faust programs on FPGA. This document explains how to install and run the toolchain v7  (version without petalinux), on a linux\footnote{tested on Ubuntu 18.04 and Ubuntu 20.04 and arch linux} machine. In practice, installing the Syfala tool-chain  means:
\begin{itemize}
\item Installing the Faust compiler, see section~\ref{faust-install} below.
\item Creating a Xilinx account and downloading/installing the 2020.2 version of the Xilinx {\tt Vivado} toolchain: {\tt vitis\_hls}, {\tt vivado} and {\tt vitis}. See section~\ref{vitis-install} below.
\item Installing Vivado board files for Digilent boards, see section~\ref{board-file-install}
\item Installing udev rules to use JTAG connection, see section~\ref{board-file-install}  
\item Cloning the Syfala directory and running a simple example as explained in Section~\ref{sec-syfala}.
\end{itemize}
Section~\ref{hard} explains the hardware configuration of the Zybo board for Syfala and Section~\ref{bug} list all the important bugs encountered when building Syfala. If you encounter a bug during the installation, please see Section~\ref{bug}.


{\bf Ubuntu dependencies:} Syfala dependencies on Linux Ubuntu are the following:\\
\texttt{sudo apt install libncurses5 libtinfo-dev g++-multilib gtk2.0}

{\bf Warning:} You need approximately 50GB of disk space to install the tool chain, and a good connection. The installation take several hours.
%If the installer prompts a choice for which version to install, select the {\bf WebPack Edition} 
           
%% {\bf Warning} all the tools of Vivado come with shell scripts that set up your {\tt \$PATH} to use them. It is quite dangerous to source them in the {\tt .bashrc} file because it provides older version of important utilities (such as {\tt cmake} for instance). We strongly advise you to use a fonction defined in your {\tt .bashrc} file such as the following:
%% ~\\

%% \begin{boxedminipage}{\textwidth}
%% \begin{verbatim}
%%   function use_vitis
%%   {
%%     source $myXilinxToolDirectory/Vivado/2020.2/settings64.sh
%%     source $myXilinxToolDirectory/Vitis_HLS/2020.2/settings64.sh
%%     source $myXilinxToolDirectory/Vitis/2020.2/settings64.sh
%%   }
%% \end{verbatim}
%% \end{boxedminipage}

\section{Installing Faust}
\label{faust-install}
It is recommanded to clone Faust from the github repository: \url{https://github.com/grame-cncm/faust}:
\begin{verbatim}
  git clone https://github.com/grame-cncm/faust faust
  cd faust
  make
  sudo make install
\end{verbatim}
If you are using an older version of Syfala, you might need to use older version of Faust (see {\tt version} files in Syfala directory). The procedure is to get the commit number of the version you need here: \url{https://github.com/grame-cncm/faust/releases}. For instance, if you use Syfala v5.4, it requires Faust version 2.31.1 (at least), it commit number is:  32a2e92c955c4e057d424ab69a84801740d37920, then execute:
\begin{verbatim}
cd faust 
git checkout  32a2e92c955c4e057d424ab69a84801740d37920
make 
sudo make install
\end{verbatim}

\section{Installing {\tt Vivado}, {\tt Vitis} and {\tt Vitis\_hls} }
\label{vitis-install}


\begin{itemize}
\item
  Open an account on https://www.xilinx.com/registration
\item
  The Xilinx download page
  (https://www.xilinx.com/support/download.html) and browse to the
  2020.2 version. The page contains links for downloading the
  ``Xilinx\_Unified\_2020.2\_1118\_1232\_Lin64.bin'' (It is available
  for both Linux and Windows but Syfala compiles only on Linux).

  \begin{itemize}
  \item
    Download the Linux installer
    \texttt{Xilinx\_Unified\_2020.2\_1118\_1232\_Lin64.bin}
  \end{itemize}
\item
  execute
  \texttt{chmod\ a+x\ Xilinx\_Unified\_2020.2\_1118\_1232\_Lin64.bin}
\item
  execute \texttt{./Xilinx\_Unified\_2020.2\_1118\_1232\_Lin64.bin}

  \begin{itemize}
  \item
    We suggest to use the ``Download Image (Install Separately)''
    option. It creates a directory with a xsetup file to execute that
    you can reuse in case of failure during the installation
  \end{itemize}
\item
  execute \texttt{./xsetup}

  \begin{itemize}
  \item
    Choose to install \textbf{Vitis} (it will still install
    \textbf{Vivado}, \textbf{Vitis}, and \textbf{Vitis HLS}).
  \item
    It will need 110GB of disk space: if you uncheck \emph{Versal ACAP} and \emph{Alveo acceleration
    platform}, it will use less space and still work.
  \item
    Agree with everything and choose a directory to install
    (e.g.~\textasciitilde/Xilinx)
  \item
    Install and wait for hours\ldots{}
  \end{itemize}
\item
  Setup a shell function allowing to use the tools when necessary (add
  this to your \texttt{\textasciitilde{}/.bashrc},
  \texttt{\textasciitilde{}/.zshrc} or whatever you're currently using,
  replacing \texttt{\$XILINX\_ROOT\_DIR} by the directory you chose to
  install all the tools)

  \begin{itemize}
  \item
\begin{verbatim}
  export XILINX_ROOT_DIR=$HOME/Xilinx
\end{verbatim}
  \end{itemize}
\end{itemize}

Then Install missing Vivado board files for Digilent boards and drivers for linux (explained in Section~\ref{board-file-install} below).

\knownbug{You HAVE to read sections~\ref{localSetting} (locale setting) and \ref{2k22patch-install} (vivado 2022 bug patch). If you do not, you might end up with unpredictible behaviour of Vivado.
}




\section{Installing Vivado Board Files and Linux drivers}
\label{board-file-install}

\subsection{Cable drivers (Linux only)}
\label{sec-udev}
\begin{itemize}
\item
  go to:\\
  \texttt{\$XILINX\_ROOT\_DIR/}\\
  \texttt{Vivado/2020.2/data/xicom/cable\_drivers/lin64/install\_script/install\_drivers}\\
  directory
\item
  run \texttt{./install\_drivers}
\item
  run \texttt{sudo\ cp\ 52-xilinx-digilent-usb.rules\ /etc/udev/rules.d}, this
  allows \textbf{JTAG} connection through \textbf{USB}.
\end{itemize}

\subsection{Vivado Board Files for Digilent Boards}
{\bf Important}: This step is needed to enable vivado to generate code for the Zybo Z10

\begin{itemize}
\item
  download:\\
\href{https://github.com/Digilent/vivado-boards/archive/master.zip?\_ga=2.76732885.1953828090.1655988025-1125947215.1655988024}{https://github.com/Digilent/vivado-boards/archive/master.zip}
\item
  Open the folder extracted from the archive and navigate to its
  \texttt{new/board\_files} folder. You will be copying all of this
  folder's subfolders
\item
  go to
  \texttt{\$XILINX\_ROOT\_DIR/Vivado/2020.2/data/boards/board\_files}
\item
  \textbf{Copy} all of the folders found in vivado-boards
  \texttt{new/board\_files} folder and \textbf{paste} them into this
  folder
\end{itemize}

%
\subsection{Installing the 2022
patch}\label{2k22patch-install}

\begin{itemize}
\item
  Follow this link:
  \href{https://support.xilinx.com/s/article/76960?language=en_US}{https ://support.xilinx.com/s/article/76960?language=en\_US}
\item
  Download the file at the bottom of th page and unzip it in
  \texttt{\$XILINX\_ROOT\_DIR}
\item
  run \texttt{cd\ \$XILINX\_ROOT\_DIR}
\item
  run (in one single command line):\\
  \texttt{export\ LD\_LIBRARY\_PATH=\$PWD/Vivado/ $\backslash$} \\
  \texttt{\ \ \ \ \ \ \ \ 2020.2/tps/lnx64/python-3.8.3/lib/ $\backslash$}\\
  \texttt{\ \ \ \ \ \ \ \ Vivado/2020.2/tps/lnx64/python-3.8.3/bin/python3\ y2k22\_patch/patch.py}
\end{itemize}




\section{Use Syfala (clone and launch)}
\label{sec-syfala}
The syfala repository is freely accessible (reading only) on  github (\url{https://github.com/inria-emeraude/syfala}), you have to have a github account of course to clone it. As mentionned before, there may be several sub-directories with different version of Syfala (i.e. different interface for Faust hardware IP). Here are the step needed to run Syfala (after having following the installation instruction of Sections above):
\begin{itemize}
\item Clone the Syfala github repository.
\item install the {\tt syfala.tcl} script
\item Run the script
\end{itemize}

\subsection{Clone the Syfala repository}
to clone the version needed and compile a first architecture you can use the following commands:\\

\begin{boxedminipage}{\textwidth}
  \begin{verbatim}
    git clone https://github.com/inria-emeraude/syfala mysyfala
    cd mysyfala/
    ./syfala.tcl install
    syfala examples/virtualAnalog.dsp
\end{verbatim}
\end{boxedminipage}

~\\

or if you have installed your ssh key on github:\\

\begin{boxedminipage}{\textwidth}
  \begin{verbatim}
    git git@github.com:inria-emeraude/syfala.git mysyfala
    cd mysyfala/
    ./syfala.tcl install
    syfala examples/virtualAnalog.dsp
\end{verbatim}
\end{boxedminipage}


\subsection{Use the {\tt syfala.tcl} script}

the command:

\texttt{\$\ ./syfala.tcl\ install}

will install a
\textbf{symlink} in \textbf{/usr/bin}. After this you'll be able to just
run:

\texttt{\$\ syfala\ myfaustprogram.dsp}

You'll also have to \textbf{edit} your shell \textbf{resource}
\textbf{file} (\textasciitilde/.\textbf{bashrc} /
\textasciitilde/.\textbf{zshrc}) and set the following environment
variable:

\begin{verbatim}
export XILINX_ROOT_DIR=/my/path/to/Xilinx/root/directory
\end{verbatim}

\texttt{XILINX\_ROOT\_DIR} is the root directory where all of the Xilinx
tools (Vivado, Vitis, Vitis\_HLS) are installed.


\subsubsection{Major Syfala commands}\label{quick-start}

\hypertarget{build-examples}{%
\paragraph{build examples}\label{build-examples}}

\begin{lstlisting}
$ syfala examples/virtualAnalog.dsp
# -> runs full toolchain on the virtualAnalog.dsp Faust dsp file, which will be ready to be flashed afterwards on a Zybo Z710 board (by default)

$ syfala examples/virtualAnalog.dsp --board GENESYS --sample-rate 96000
# -> runs full toolchain for the Genesys board, with a sample-rate of 96000Hz

$ syfala examples/phasor.dsp --export phasor-build
# -> runs full toolchain on 'phasor.dsp', automatically exporting the build to 
# export/phasor-build.zip

$ syfala examples/fm.dsp --arch --hls --report
# -> only run 'arch' & 'high-level synthesis' (HLS) step on 'fm.dsp', and show the report afterwards.

$ syfala examples/fm.dsp --board Z20 --arch --hls --export z20-fm-hls-build
# -> only run 'arch' & HLS step on 'fm.dsp' for Zybo Z20 board, and export the build.
\end{lstlisting}

\subsubsection{Additional Syfala `one-shot' commands}

\begin{tabular}{|c|p{9cm}|c|}
  \toprule
  name & description & arguments \\
\midrule
\texttt{install} & installs this script as a symlink in /usr/bin/ &
none \\
\texttt{clean} & deletes current build directory & none \\
\texttt{import} & sets previously exported build as the
current build & .zip target\\
\texttt{export} & exports current build in a .zip file located in the
`export' directory & build name\\
\texttt{report} & prints HLS report of the current build & none \\
\texttt{demo} & fully builds demo based on default example
(virtualAnalog.dsp) & none \\
\texttt{flash} & flashes current build onto target device & none \\
\texttt{gui} & executes the Faust-generated gui application & none \\
\texttt{rebuild-app} & rebuilds the host control application, without
re-synthesizing the whole project & none \\
\texttt{open-project} & opens the generated .xpr project
with Vivado & none \\

\bottomrule
\end{tabular}

\paragraph{Syfala `one-shot' command examples}

\begin{verbatim}
$ syfala clean
$ syfala demo
$ syfala export my-current-build
$ syfala rebuild-app
$ syfala flash
\end{verbatim}

\subsubsection{General Options to Syfala command}

\begin{tabular}{|c|c|p{8cm}|}
  \toprule
option & accepted values & description \\
\midrule
\texttt{-c\ -\/-compiler} & \texttt{HLS\ -\ VHDL} & chooses between
Vitis HLS and faust2vhdl for DSP IP generation. \\
\texttt{-\/-reset} & / & resets current build directory before building
(\textbf{careful}! all files from previous build will be lost) \\
\bottomrule
\end{tabular}

\subsubsection{Controling Syfala Run steps}

\textbf{Note}: the \texttt{-\/-all} is not necessary if you wish to run
all steps, just run:

\texttt{syfala\ myfaustdsp.dsp}

\begin{tabular}{|c|p{12cm}|}
  \toprule
\texttt{-\/-all} & runs all toolchain compilation steps (from
\texttt{-\/-arch} to \texttt{-\/-gui}) \\
\midrule
\texttt{-\/-arch} & uses Faust to generate ip/host cpp files for HLS and
Host application compilation \\
\texttt{-\/-hls\ -\/-ip} & runs Vitis HLS on generated ip cpp file \\
\texttt{-\/-project} & generates Vivado project \\
\texttt{-\/-synth} & synthesizes full Vivado project \\
\texttt{-\/-host\ -\/-app} & compiles Host application, exports sources
and .elf output to \texttt{build/sw\_export} \\
\texttt{-\/-gui} & compiles Faust GUI controller \\
\texttt{-\/-flash} & flashes boot files on device at the end of the
run \\
\texttt{-\/-report} & prints HLS report at the end of the run \\
\texttt{-\/-export} & \texttt{\textless{}id\textgreater{}} exports build
to export/ directory at the end of the run \\
\bottomrule
\end{tabular}

\subsubsection{Controlling the architecture build by Syfala}

\begin{tabular}{|c|c|c|}
  \toprule
parameter & accepted values & default value \\
\midrule
\texttt{-\/-memory,\ -m} & \texttt{DDR\ -\ STATIC} & \texttt{DDR} \\
\texttt{-\/-board,\ -b} & \texttt{Z10\ -\ Z20\ -\ GENESYS} &
\texttt{Z10} \\
\texttt{-\/-sample-rate} &
\texttt{48000\ -\ 96000\ -\ 192000\ -\ 384000\ -\ 768000} &
\texttt{48000} \\
\texttt{-\/-sample-width} & \texttt{16\ -\ 24\ -\ 32} & \texttt{24} \\
\texttt{-\/-controller-type} &
\texttt{DEMO\ -\ PCB1\ -\ PCB2\ -\ PCB3\ -\ PCB4} & \texttt{PCB1} \\
\texttt{-\/-ssm-volume} & \texttt{FULL\ -\ HEADPHONE\ -\ DEFAULT} &
\texttt{DEFAULT} \\
\texttt{-\/-ssm-speed} & \texttt{FAST\ -\ DEFAULT} & \texttt{DEFAULT} \\
\bottomrule
\end{tabular}
\\

Here is the description of these parameters:\\
\begin{tabular}{|c|p{12cm}|}
  \toprule
parameter & description \\
\midrule
\texttt{-\/-memory,\ -m} & selects if \textbf{external} \textbf{DDR3} is
used. Enable if you use some delay, disable if you do want any memory
access (should not be disabled) \\
\texttt{-\/-board} & Defines target board. \textbf{Z10} ,\textbf{Z20}
and \textbf{GENESYS} only. If you have a VGA port (rather than 2 HDMI
ports), you have an old Zybo version, which is not supported. \\
\texttt{-\/-sample-rate} & Changes \textbf{sample rate} value (Hz). Only
48kHz and 96kHz is available for \textbf{SSM} embeded codec. 192000
(\textbf{ADAU1777} and \textbf{ADAU1787} only) 384000 (\textbf{ADAU1787}
only) 768000 (\textbf{ADAU1787} only and with
\texttt{-\/-sample-\/-width\ 16} only) \\
\texttt{-\/-sample-width} & Defines \textbf{sample bit depth}
(16\textbar24\textbar32) \\
\texttt{-\/-controller-type} & Defines the controller used to drive the
controls when \textbf{SW3} is \textbf{UP}. (\textbf{SW3} \textbf{DOWN}
for \textbf{software} control), \textbf{SEE BELOW} for details on each
value \\
\texttt{-\/-ssm-volume} & Chooses audio codec to use. For now, it only
changes the scale factor. \textbf{FULL}: Maximum (\textbf{WARNING}: for
speaker only, do not use with headphones). \textbf{HEADPHONE}: Lower
volume for headphone use. \textbf{DEFAULT}: Default value +1dB because
the true 0dB (\texttt{0b001111001}) decreases the signal a little
bit. \\
\texttt{-\/-ssm-speed} & Changes \textbf{SSM ADC/DAC} sample rate.
\textbf{DEFAULT}: 48kHz sample rate. \textbf{FAST}: 96Khz sample rate \\
\bottomrule
\end{tabular}

\section{Hardware configuration (Zybo Z7-10/20)}
\label{hard}
\begin{itemize}

\item
  Jumper \textbf{JP5} should be on \emph{JTAG}
\item
  \textbf{Power select} jumper should be on \emph{USB}\\
\item
  \textbf{Switches} SW0, SW1, SW2, SW3 should be \textbf{down} (i.e. toward the opposite side of the ethernet connector\\
\item
  The \textbf{audio input} is \textbf{LINE IN} (blue), not MIC IN\\
\item
  The \textbf{audio output} is the black \textbf{HPH OUT} jack
\end{itemize}

\subsection{Control of the Syfala IP}
\label{control}

To control your DSP, you can either use a Syfala Hardware Controller Board or a
GUI on your computer. Beguinner should use GUI control.

\hypertarget{gui-sw3-down}{%
\paragraph{GUI (SW3 DOWN)}\label{gui-sw3-down}}

\textbf{SW3} should be \textbf{down} (0).

If you use GUI, open the GUIcontroller after booting with the following
command:

\begin{verbatim}
make gui
\end{verbatim}

\hypertarget{syfala-hardware-controller-board-sw3-up}{%
\paragraph{Syfala Hardware Controller Board (SW3
UP)}\label{syfala-hardware-controller-board-sw3-up}}

\textbf{SW3} should be \textbf{up} .

If you use a Hardware Controller Board, please set the
\texttt{-\/-controller-type} command-line parameter to the proper value
(see below)

\hypertarget{controller-type-values-description}{%
\subparagraph{Controller-type values
description}\label{controller-type-values-description}}

\begin{itemize}

\item
  \textbf{DEMO}: Popophone demo box
\item
  \textbf{PCB1}: Emeraude PCB config 1: 4 knobs, 2 switches, 2 sliders
  (default)
\item
  \textbf{PCB2}: Emeraude PCB config 2: 8 knobs
\item
  \textbf{PCB3}: Emeraude PCB config 3: 4 knobs, 4 switches
\item
  \textbf{PCB4}: Emeraude PCB config 4: 4 knobs above, 4 switches below
\end{itemize}

You can swap from hardware to software controller during DSP execution
by changing SW3.

\hypertarget{switch-description}{%
\subsubsection{Switch description}\label{switch-description}}

\begin{verbatim}
  SW3   SW2    SW1    SW0
+-----+-----+-------+------+
| Hard| ADAU| BYPASS| MUTE |
|     |     |       |      |
|     |     |       |      |
| GUI | SSM |USE DSP|UNMUTE|
+-----+-----+-------+------+
\end{verbatim}
\begin{itemize}

\item
  \textbf{SW3}: Controller type select: hardware (Controller board) or
  software (GUI). Default: {\bf GUI}
\item
  \textbf{SW2}: Audio codec input select (ADAU=external or SSM=onboard).
  Does not affect output. Default: \textbf{SSM}
\item
  \textbf{SW1}: Bypass audio dsp. Default: \textbf{USE DSP}
\item
  \textbf{SW0}: Mute. Default: \textbf{UNMUTE}
\end{itemize}

\hypertarget{status-leds}{%
\subsubsection{Status LEDs}\label{status-leds}}

The RGB led indicate the program state:

\begin{itemize}

\item
  \textbf{BLUE} = WAITING
\item
  \textbf{GREEN} = ALL GOOD
\item
  \textbf{ORANGE} = WARNING (Bypass or mute enable)
\item
  \textbf{RED} = ERROR (Config failed or incompatible). Could happen if
  you select SSM codec with incompatible sample rate.
\end{itemize}

The 4 LEDs above the switches indicate the switches state. If one of
them blink, it indicates the source of the warning/error.

\hypertarget{sd-card-files}{%
\subsubsection{SD card files}\label{sd-card-files}}

You can put the program on an SD card (if you want something
reproductible and easily launchable, for the demos\ldots).\\
After a \texttt{make} command, you should see a \texttt{BOOT.bin} file
in SW\_export (or you can build it with \texttt{make\ boot\_file}).\\
Put the file on the root of SD card. And don't forget to put JP5 on `SD'
position !




\section{Known bugs: Important ``tricks'' to be known!!}
\label{bug}

This section regroups all the tricks that can result in unlimited waste of time if not known. These {\em known bugs} have been kept as they have been initially written, even if some of them do not occur anymore in more recent tool version.

\subsection{Locale setting on linux}
\label{localSetting}
\knownbug{it is a known bug that {\tt vivado} is sensible to the ``locale'' environment variable on linux, hence you have to set these variables in your {\tt .bashrc} file:\\
\tt export LC\_ALL=en\_US.UTF-8\\
export LC\_NUMERIC=en\_US.UTF-8
}

If you do not, you might end up with unpredictible behaviour of Vivado.

\subsection{Patch 2022 date bug}
\label{2k22patch}
\knownbug{Vivado and Vitis tools that use HLS in the background are also affected by this issue. HLS tools set the ip\_version in the format YYMMDDHHMM and this value is accessed as a signed integer (32-bit) that causes an overflow and generates the errors below (or something similar).}

Follow this link: \url{https://support.xilinx.com/s/article/76960?language=en_US}

Download the file at the bottom of the page and unzip it in your Xilinx base install directory (Xilinx file where you have your Vitis,Vitis\_HLS and Vivado files). 

DONT FOLLOW THE README... Just check the "Known Issues:" section on the Xilinx page which takes over the readme.

From the Xilinx directory, run:
\begin{itemize}
\item export LD\_LIBRARY\_PATH=\$PWD/Vivado/2020.2/tps/lnx64/python-3.8.3/lib/
\item Vivado/2020.2/tps/lnx64/python-3.8.3/bin/python3 y2k22\_patch/patch.py
\end{itemize}

\subsection{Save the Vivado Install file in case of installation failure}
\label{installSave}

Vivado installation tends to fail. To avoid having to redownload the installation file each time you try , we suggest to use the “ Download Image (Install Separately)” option. It creates a directory with a xsetup file to execute for installing. But don't forget to duplicate the installation file, because Vivado will delete the xsetup installation file you use if you choose to let him delete all files after the installation failed.
%Oui alors c'est pas clair....
\subsection{Vivado Installation stuck at "final processing: Generating installed device list"}
If the install of Vivado is stuck at "final processing: Generating installed device list", cancel it and install the libncurses5 lib:
\begin{verbatim}
sudo apt install libncurses5
\end{verbatim}

\subsection{Installing Vivado Board Files for Digilent Boards}
\label{boardfiles}
It is necessary, once Vivado install, to add support for new digilent board.
the content of directory {\tt board\_files } has to be copied in \verb#$vivado/2019.2/data/boards/board_files#
(see \begin{verbatim}https://reference.digilentinc.com/learn/programmable-logic/tutorials/\ 
    zybo-getting-started-with-zynq/start?redirect=1#
\end{verbatim}

Or directly here: \url{https://github.com/Digilent/vivado-boards}

\subsection{Cable drivers (Linux only)}
For the Board to be recognized by the Linux system, it is necessary to install additional drivers. See \url{https://digilent.com/reference/programmable-logic/guides/install-cable-drivers}


\subsection{Digilent driver for linux}
On some linux install, programming the Zybo board will need to install an additionnal ``driver'': Adept2 \url{https://reference.digilentinc.com/reference/software/adept/start?redirect=1#software_downloads}

\subsection{Vitis installation}
{\bf Warning} Apparently the installation process does not end correctly if the {\tt libtinfo-dev} package is not correctly installed (\url{https://forums.xilinx.com/t5/Installation-and-Licensing/Installation-of-Vivado-2020-2-on-Ubuntu-20-04/td-p/1185285}. In case of doubt, execute these commands (april 2020):
\begin{verbatim}
sudo apt update
sudo apt install libtinfo-dev
sudo ln -s /lib/x86_64-linux-gnu/libtinfo.so.6 /lib/x86_64-linux-gnu/libtinfo.so.5
\end{verbatim}

\subsection{"'sys/cdefs.h' file not found" during vitis\_HLS compilation}
If Vitis HLS synthesis fails with the following error:
\begin{verbatim}
'sys/cdefs.h' file not found: /usr/include/features.h 
\end{verbatim}
You have to install the g++-multilib lib
\begin{verbatim}
sudo apt-get install g++-multilib
\end{verbatim}

\subsection{Board files: version 1.0 or 1.1?}
Digilent updated his board file repository (mentioned above in section~\ref{boardfiles}) and unfortunately changes the version of the board from 1.0 to 1.1. This change must be reverted because it is not taken into account in past version of vivado.

It you have a message like:
\begin{verbatim}
source /home/romain/reps/syfala/build/sources/project.tcl -notrace
ERROR: [Board 49-71] The board_part definition was not found for
  digilentinc.com:zybo-z7-10:part0:1.0. The project's board_part property was 
 not set, but the project's part property was set to xc7z010clg400-1. 
 Valid board_part values can be retrieved with the 'get_board_parts'
 Tcl command. Check if board.repoPaths parameter is set and the board_part 
 is installed from the tcl app store.
\end{verbatim}

You should do the following:
\begin{itemize}
  \item 
    go into directory:\\
    {\tt Vivado/2020.2/data/boards/board\_files/zybo-z7-10/A.0}
\item Edit the file {\tt 'board.xml'}
  and change\\
  {\tt <file\_version>1.1</file\_version>}\\ into\\ {\tt  <file\_version>1.0</file\_version>}
\item (Same thing for Z20 if you use Z20).
\end{itemize}


\section{The syfala team}
\label{team}
Here is a list of person that have contributed to the Syfala project:
\begin{itemize}
\item Tanguy Risset
\item Yann Orlarey
\item Romain Michon
\item Stephane Letz
\item Florent de Dinechin
\item Alain Darte
\item Yohan Uguen
\item Gero Müller
\item Adeyemi Gbadamosi
\item Ousmane Touat
\item Luc Forget
\item Antonin Dudermel
\item Maxime Popoff
\item Thomas Delmas
\item Oussama Bouksim
\item Pierre Cochard
\end{itemize}


